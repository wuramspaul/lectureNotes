\documentclass[14pt]{extarticle}
\usepackage[russian]{babel}
\usepackage[utf8x]{inputenc}
\usepackage{amsmath}
\usepackage{graphicx}
\usepackage{indentfirst}
\renewcommand{\labelenumii}{\arabic{enumi}.\arabic{enumii}.}
\begin{document}
	\section{История государства и права}
	\textbf{Государство} --- это политика--территориальная организация государственного образования, общества имеющая определённый аппарат власти
	
	\textbf{\textit{Право}} --- это совокупность правовых норм регулирующих общественное отношения.
	
	\textbf{\textit{АТЕ}} --- Административно Территориальная Единица
	
	\begin{center}
		\textbf{\textit{Форма государства}}
	\end{center}
	\begin{enumerate}
		\item Форма правления -- организация органов гос власти
			\subitem Монархия (возглавляет монарх)
			\subitem Республика (источник власти - народ)
		\item Форма государственного устройства -- организация территориального деления
			\subitem Унитарное государство
			\subitem Федерация 
			\subitem Конфедерация
		\item Политико-правовой режим -- совокупность методов и средств с помощью которых осуществляется государственная власть
			\subitem Демократический
			\subitem Недемократический
	\end{enumerate}  
	
	\newpage	
	\begin{center}
	\textbf{\textit{Источники права}}
	\end{center}
	\begin{enumerate}
		\item Нормативно правовые акты (НПА): \textit{Конституция, Ратифицированные международные договоры, декреты и указы презедента, законы в том числе кодексы, постановления}
		\item Правовой обычай: \textit{практика государств принятые ими}
		\item Юридический претендент
		\item Нормативный договор
	\end{enumerate}
	\section{Права человека -- высшая ценность общества}
	Права и свободы человека делятся на:
	\begin{enumerate}
		\item Личные права
		\begin{enumerate}
			\item На жизнь (является главным не отемлемым правом)
			\item Право на свободу, неприкосновенность и достоинство личности
			\item Неприкосновенность жилища
			\item Свобода передвижения и выбора места жительства
			\item Право на определение к религии
			\item Свобода мнений и убеждений
			\item Право на судебную защиту
			\item Право на национальную принадлежность, пользование родным языком
			\item Право на юридическую помощь
			\item Право на защиту международных организаций
		\end{enumerate}
		\item Политические права (относятся только к гражданам гос--ва)
		\begin{enumerate}
			\item Право на упровение далами в обществе и государства
			\item Право избирать и быть избранным
			\item Право на обращение в гос ограны
			\item Право на свободу объединения
			\item Свобода собраний, митингов, уличных шествий и фемонстваций
			\item Право на получение, хранения и размещениея информации о деятельности гос. органов, общ. обединений о состоянии окружающей среды и прочее
			\item Право на равный доступ к любым должностям в гос. органах
		\end{enumerate}
		\item Экономические, социальные и культурные права
		\begin{enumerate}
			\item Право на труд
			\item Право на справедливую долю вознаграждения 
			\item Право трудящихся на отдых
			\item Право собственности
			\item Право на жилище 
			\item Право на охрану здоровья
			\item Право на благоприятную окружающию среду
			\item Право на соц обеспечения в случаи болезни, потери кормилца, старости
			\item Право на участие в культурной жизни
		\end{enumerate}
	\end{enumerate}
\end{document}