\documentclass[14pt]{extarticle}
\usepackage[russian]{babel}
\usepackage[utf8x]{inputenc}
\usepackage{amsmath}
\usepackage{graphicx}
\usepackage{indentfirst}

\begin{document}
	\section{История государства и права}
	\textbf{Государство} --- это политика--территориальная организация государственного образования, общества имеющая определённый аппарат власти
	
	\textbf{\textit{Право}} --- это совокупность правовых норм регулирующих общественное отношения.
	
	\textbf{\textit{АТЕ}} --- Административно Территориальная Единица
	
	\begin{center}
		\textbf{\textit{Форма государства}}
	\end{center}
	\begin{enumerate}
		\item Форма правления -- организация органов гос власти
			\subitem Монархия (возглавляет монарх)
			\subitem Республика (источник власти - народ)
		\item Форма государственного устройства -- организация территориального деления
			\subitem Унитарное государство
			\subitem Федерация 
			\subitem Конфедерация
		\item Политико-правовой режим -- совокупность методов и средств с помощью которых осуществляется государственная власть
			\subitem Демократический
			\subitem Недемократический
	\end{enumerate}  
	
	\newpage	
	\begin{center}
	\textbf{\textit{Источники права}}
	\end{center}
	\begin{enumerate}
		\item Нормативно правовые акты (НПА): \textit{Конституция, Ратифицированные международные договоры, декреты и указы презедента, законы в том числе кодексы, постановления}
		\item Правовой обычай: \textit{практика государств принятые ими}
		\item Юридический претендент
		\item Нормативный договор
	\end{enumerate}
\end{document}