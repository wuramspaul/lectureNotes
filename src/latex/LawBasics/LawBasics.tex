\documentclass[14pt]{extarticle}
\usepackage[russian]{babel}
\usepackage[utf8x]{inputenc}
\usepackage{amsmath}
\usepackage{graphicx}
\usepackage{indentfirst}
\renewcommand{\labelenumii}{\arabic{enumi}.\arabic{enumii}.}
\begin{document}
	\section{История государства и права}
	\textbf{Государство} --- это политика--территориальная организация государственного образования, общества имеющая определённый аппарат власти
	
	\textbf{\textit{Право}} --- это совокупность правовых норм регулирующих общественное отношения.
	
	\textbf{\textit{АТЕ}} --- Административно Территориальная Единица
	
	\begin{center}
		\textbf{\textit{Форма государства}}
	\end{center}
	\begin{enumerate}
		\item Форма правления -- организация органов гос власти
			\subitem Монархия (возглавляет монарх)
			\subitem Республика (источник власти - народ)
		\item Форма государственного устройства -- организация территориального деления
			\subitem Унитарное государство
			\subitem Федерация 
			\subitem Конфедерация
		\item Политико-правовой режим -- совокупность методов и средств с помощью которых осуществляется государственная власть
			\subitem Демократический
			\subitem Недемократический
	\end{enumerate}  
	
	\newpage	
	\begin{center}
	\textbf{\textit{Источники права}}
	\end{center}
	\begin{enumerate}
		\item Нормативно правовые акты (НПА): \textit{Конституция, Ратифицированные международные договоры, декреты и указы презедента, законы в том числе кодексы, постановления}
		\item Правовой обычай: \textit{практика государств принятые ими}
		\item Юридический претендент
		\item Нормативный договор
	\end{enumerate}
	
	\newpage
	\section{Права человека -- высшая ценность общества}
	Права и свободы человека делятся на:
	\begin{enumerate}
		\item Личные права
		\begin{enumerate}
			\item На жизнь (является главным не отемлемым правом)
			\item Право на свободу, неприкосновенность и достоинство личности
			\item Неприкосновенность жилища
			\item Свобода передвижения и выбора места жительства
			\item Право на определение к религии
			\item Свобода мнений и убеждений
			\item Право на судебную защиту
			\item Право на национальную принадлежность, пользование родным языком
			\item Право на юридическую помощь
			\item Право на защиту международных организаций
		\end{enumerate}
		\item Политические права (относятся только к гражданам гос--ва)
		\begin{enumerate}
			\item Право на упровение далами в обществе и государства
			\item Право избирать и быть избранным
			\item Право на обращение в гос ограны
			\item Право на свободу объединения
			\item Свобода собраний, митингов, уличных шествий и фемонстваций
			\item Право на получение, хранения и размещениея информации о деятельности гос. органов, общ. обединений о состоянии окружающей среды и прочее
			\item Право на равный доступ к любым должностям в гос. органах
		\end{enumerate}
		\item Экономические, социальные и культурные права
		\begin{enumerate}
			\item Право на труд
			\item Право на справедливую долю вознаграждения 
			\item Право трудящихся на отдых
			\item Право собственности
			\item Право на жилище 
			\item Право на охрану здоровья
			\item Право на благоприятную окружающию среду
			\item Право на соц обеспечения в случаи болезни, потери кормилца, старости
			\item Право на участие в культурной жизни
		\end{enumerate}
	\end{enumerate}

	\newpage
	\section{Основы констуционного права}
	\textbf{\textit{Конституция}} --- это основной закон государства, обладающей высшей юридической силой и закрепляющий осново пологающие принципы и нормы регулирующие выжнейшее общественное отношение.
	Юридические свойства конституции: 
	\begin{enumerate}
		\item Учредительный характер
		\item Конституция является основной закон гос--ва
		\item Высшая юридическая сила
	\end{enumerate}
	
	Призедент Республики Беларусь является \textit{Главной гос--ва},гарантом Конституции РБ, прав и свобод человека и гражданина.
	
	Президентом РБ может быть избран только гражданин РБ по рождению, не моложе 35 лет, обладающий избирательным правом и постоянно проживающий в РБ не менее 10 лет непосредственно перед выборами.
	
	Призедент РБ:
	\begin{enumerate}
		\item Назначает республиканские референдумы
		\item Назначает очередные и внеочередные выборы а Палату председателей, Совета Республики и местные представительные органы.
		\item Распускает палаты в случаях и в порядке, предусмотренных Конституцией.
		\item С согласия Палаты представителей назначает на должность Примьер--министра
		\item Осуществляет иные полномочия
	\end{enumerate}
	\subsection{Законадательная ветвь власти}
	
	Представляет парламент --- \textit{Национальное Собрание Республики Беларусь}
	
	Парламент --- Национальное собрание РБ является законодательным органом РБ.
	
	Парламент состоит из двух палат --- Палаты представителей и Совета Республики. Состав палаты представителей -- 110 депутатов. и 64 члена Совета республики
	
	Дипутат Палаты представитей может быть гражданин РБ, достигший 21 года.
	
	Членом Совета Республики млжет быть гражданин РБ, достигший 30 лет и проживающий на територии соотведстующей облости, города Минста не мения 5 лет.
	
	\textbf{Палата представителей}
	\begin{enumerate}
		\item Рассмотрение проекты законом.
		\item Назначение выбора Призедент.
		\item Даёт согласие Призедент на назначение Примьер--министра.
		\item Рассматривает по инициативе Примьер--министра вопрос о доверии к Правительству.
	\end{enumerate}

	\textbf{Совет Республики}
	\begin{enumerate}
		\item Одобряет и отклоняет принятые Палатой представителей проекты законов о внесении и дополнении в Конституцию; о толковании Конституции; Проекты иных законов
		\item Избирает шесть судей Конституционного суда
		\item Отменяет шесть членов Центральной комиссии РБ по выбору и проведению республиканских референдумов.
		\item отменяет решения местных Советов депутатов, не соответствующий законодательству.
	\end{enumerate}

	Исполнительную власть в РБ осуществляет Правительство --- Совет Министров Республики Беларусь --- центральный орган государственного управления. 
	Правительство РБ состоит из Примьер---министра, его заместителей и министров.
	
	\textbf{Правительство Республики Беларусь}
	\begin{enumerate}
		\item руководит системой подчинных его огранов государственного управления и других органов исполнительной власти.
		\item разрабатывает основные направления внутренней и внешней политики и принимает меры по из реализации
		\item разрабатывает и представляет Президенту для внесения в Парламент проект республиканского бюджета и отчёт по его исполнению
		\item обеспечивает исполнения Конституции, законов и декретов, указов и распоряжения Президента.
	\end{enumerate}
	
	Судебная власть в Республики Беларусь принадлежит судам. Система судов строится на принципах территориальности и специализации. Судоустройство в Республики Беларусь определяется законом. 
	
	Образование чрезвычайных судов запрещается.
	
	Судебная система Республики Беларусь:
	\begin{enumerate}
		\item Конституционный суд Республики Беларусь
		\item Суду общей юрисдикции. (Осуществляют правосудье посредствам гражданского, уголовного, административного судопроизводства и судопроизводство по экономическим делам.)
	\end{enumerate}
	Система судов общей юрисдикции строится на принципах территориальности и специализации.
	
	\subsection{Контрольно надзорные органы}
	
	Надзор за точным и единообразным исполнением законов, декретов, указов и иных нормативных актов минестерства и другими подведомстенными Совету Министров органами, местными представительными и исполнительными органами, предприятиями, организациями и исполнительными органами, предпреятиями, организациями и учереждениями, общественными, должностными и учреджденияси, общественными объединениями, должосными лицами и гражданами возлагается на Генерального прокурора Республики Беларусь и подчинённых ему прокуроров.
	
	Государственный контроль за исполнением республиканского буджета, исполнением актов Призедента, Парламента, Правительства и других государственных органов, регулирующих отношения государственной собственности, хозяйственные, финансовые и нологовые отношения, осуществляет Комитет государственного контроля.
\end{document}