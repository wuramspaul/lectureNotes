\documentclass[a5paper]{article}
\usepackage[russian]{babel}
\usepackage[utf8x]{inputenc}
\usepackage{amsmath}
\usepackage{graphicx}
\usepackage{indentfirst}


\begin{document}
	\section{Введение в теорию вероятности}
	
	\textbf{\textit{Теория вероятности}} --- раздел математики изучающие случайные события, величины их свойства и операции над ними

	Самые ранние работы по теории вероятности относятся к 17 в. Важный вклад в теорию вероятности внёс \textit{Якоб Бернули}, он дал доказательство \textbf{закона больших чисел в простейшим случаи производимых испытаний}
	
	В первой половине 19 в. теория вероятности начинает применятся к анализу ошибок поведения, во второй половине 19 в. были доказаны \textbf{законы больших чисел }, \textbf{центрально предельная теорема} а также разработана \textbf{теория цепей}. 
	
	Современный вид теория вероятности получила благодаря аксиоматизации предложений \textit{Андрея Николаевичем Колмогоровым} 
	
	В результате теория вероятности пробрела строгий математический вид и окончательно стала восприниматься как один из разделов математики
	
	\newpage
	\section{Испытания и события. Виды испытаний событий. Операции объединения и пересечения событий, их свойства}  	
	\subsection{Испытания и события}

	\textbf{\textit{Испытания}} --- осуществление некоторого комплекса условий, в результате которого непременно произойдёт какое либо событие. 

	\textbf{\textit{Случайное событие}} --- событие которое может произойти или не произойти в результате испытания.
	
	\textbf{\textit{Достоверное событие}} --- событие, которое обязательно произойдёт в результате данного испытания
	
	\textbf{\textit{Невозможные события}} --- события которые некогда не произойдут в результате данного испытания.
	
	\textbf{\textit{Несовместимые события}} --- события, которые не могут появится одновременно в результате данного испытания
	
	Если событие могут произойти одновременно, то они называются \textbf{совместимыми}.
	
	\textbf{\textit{Равновозможные события}} --- события которые имеют одинаковый шанс произойти в результате данного испытания
	
	Множество, элементами --- которого являются все несовместимые, равновозможные исходы данного испытания, называют \textbf{\textit{пространством элементарных исходов}}.
	
	\newpage
	\subsection{Классическое определение вероятности}
	
	\textbf{\textit{Вероятностью события}}, называется отношения числа элементарных исходов благоприядстующиму данному событию(m) к числу всех равновозможных исходов опыта, в котором может появится это событие.
	
	\begin{itshape}
		Вероятность события \textbf{А} обозначают обозначают \textbf{P(A)}, здесь P -- первая буква французского слова Probability (пер. случайность). В соатведствии с определением 
		\[ P(A) = \dfrac{m}{n} \]  
		
		m -- число элементарных исходов, благоприядствующих событию \textbf{A}
		
		n -- число всех равновозможных элементарных исходов опыта образующие полную группу событий.
	\end{itshape}

	Это определение называется \textbf{\textit{классическим}} она возникла на начальном этапе развития теории вероятности.
	
	Вероятность событий имеет следующие \textbf{свойства} 
	\begin{enumerate}
		\item Вероятность достоверного события равна 1. Обозначим достоверное событие буквой \textbf{U}. Для достоверного события m = n, поэтому 
		\[ P(U) = 1 \]
		\item Вероятность невозможного события равна 0. Обозначим невозможное событие буквой \textbf{V}. Для невозможных событий m = 0, поэтому 
		\[ P(V) = 0 \]
		\item Вероятность случайного события выражается положительным числом меньше 1. Поскольку для события \textbf{А} выполняется неравенство:
		\[ 0 < m < n \]
		или 
		\[0 < \dfrac{m}{n} < 1 \]
		то,
		 \[  0 < P(A) < 1 \]
		\item Вероятность любого случайного события \textbf{В} удостоверяет неравенство 
		\[ 0\leq P(B)\leq 1  \]
	\end{enumerate}	
	
	\vspace{\baselineskip}
	\begin{center}
		\textbf{Пример решения задачи}
	\end{center} 
	\par\textbf{Задача.} В урне 10 одинаковых по массе и размеру шаров из которых 6 голубых и 4 красных. Из урны извлекают один шар. Какова вероятность того, что извлеченный шар окажется голубым 
	
	\textbf{Решение.} Событие извлечение шара голубым является событием. Пусть это событие называется \textbf{А}. Дальнейшие испытание имеет 10 элементарных исходов из которых 6 является благоприятными. 
	\[ P(A) = \dfrac{6}{10} = 0,6 = 60\% \]
	\par \textbf{Ответ:} Вероятность изъятия шара голубого цвета равна 60\%. 
	
	\newpage
	\section{Комбинаторика и вероятность. Статистическое определение вероятности. Геометрические вероятности}
	
	\textbf{Комбинаторика} изучает, способы подсчёта числа элементов в конечных множествах
	
	Формулы комбинаторики используют при непосредственном вычислении вероятности.
	\subsection{Перестановки}
	Множество элементов состоящих из одних и  тех--же различных элементов и отличающихся друг от друга только их порядком называются перестановками этих элементов
	
	Число всевозможных перестановок из n элементов обозначается через $P_{n} $--- это число равно n!.
	\[ P_{n} = n!\]
	
	Где n! --- это произведение последовательно умноженных чисел от 1 до n. 
	
	Это используют для конечного множества
	
	Для пустого множества факториал равен 1.
	\[
		0! = 1
	\]
	
	\subsection{Размещения}
	Размещением называется множество составленным из\textit{ n различных элементов по m элементов}, которые отличаются либо составом элементов либо порядком элемента
	Число всех возможных размещений определяется формулой: 
	\[ A^{m}_{n}  = n(n-1)(n-2) ...(n-m+1)\] 
	
	\subsection{Сочетания}
	Сочетаниями из n различных элементов по m, называется множеством содержащих m элементов из числа n заданных, и которые отличаются хотя бы одним элементом. Число сочитаний n элементов по m обозначается $ {C}^{m}_{n} $ и выражается следующим образом:
	\[
		{C}^{m}_{n} = \dfrac{n!}{m!(n-m)!}	
	\]  
	
	По переведённым формулам предполагают, что 
	\[
	{C}^{0}_{n} = 1
	\]  
	
	Отметим что число перестановок, размещений и сочетаний связаны равенством
	\[
		{C}^{m}_{n} = \dfrac{{A}^{m}_{n}}{{P}_{m}}
	\] 
	
	Классическое определение вероятности предполагает, что все элементарные исходы равновозможные.
	
	\subsection{Частота}
	Относительной частотой событий называют отношение числа опытов, в которах появились эти события к числу всех произведённых опытов, и обознается:
	\[
		W(A) = \dfrac{m}{n}
	\]
	
	m --- число опытов где проявилось событие, n --- число произведённых опытов.
	
	\begin{itshape}
		Вероятность события называется число, около которого группируется значение частоты данного события в различных сериях большого числа испытаний.
	\end{itshape}

	Эти определение вероятности называется \textbf{статистическим определением теории вероятности}.
\end{document}