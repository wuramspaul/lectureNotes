\documentclass[a4paper,14pt]{extarticle}
\usepackage[russian]{babel}
\usepackage[utf8x]{inputenc}
\usepackage{indentfirst}

\title{\textbf{ЗащитаА компьютерной информации}}
\author{Михальцов П.А.}

\begin{document}
	\maketitle
	\section{Защита информации в информационно вычислительных системах}
	\subsection{Проблемы защиты компьютерной информации}
	Вопросы 
	\begin{enumerate}
		\item Основное понятия об угрозах информационной безопасности
		\item Актуальность проблемы обеспечения информационной безопасности. Задачи защиты информации
		\item Задачи информационной безопасности
	\end{enumerate}
	\subsubsection{Основное понятия об угрозах информационной безопасности}
	\textbf{\textit{Информационная безопасность}} --- такое состояние системы, при которой она может противостоять против дестабилизирующий воздействию внешних и внутрений угроз, а также -- её функционирования не создаёт информационных угроз для элементов самой системы и внешней среды
	
	Обеспечение информационной безопасности проблемы может быть достигнуто лишь при взаимоувязанном решении трёх составляющих проблем:
	\begin{itemize}
		\item защита находящееся в системе информации от дестабилизирующего воздействия внешних и внутренних угроз информации
		\item защита элементов системы от дестабилизирующий воздействия внешних и внутренних информационных угроз 
		\item защита внешний среды от информационной угроз со стороны рассматриваемой системы 
	\end{itemize}

	\vspace{\baselineskip}
	Под \textbf{\textit{защитой информации}} понимать совокупность мероприятий и действий, направленнох на обеспечение её безопасности -- конфендициальность и целосность -- в процессе сбора, перердачи, обработки и хранения.
	
	\textbf{\textit{Безопасность информациии}} - это свойство (состояние) передаваемой накапливаемой, обрабатываемой и хранимой информации, характеризующие её степень защищенности от дистабилизирующиго воздействия внешней среды и внутренних угроз то есть её конфендициальность, сигнальная скрытность(энергетическая и структурная) и целостность -- устойчивость к разрушающим, имитирующими и искажающим воздействиям и помехам.	
	
	Под \textbf{\textit{защитой информации}} в более широком смысле понимают комплекс организационных, правовых и технических мер по предотвращению угроз информационной безопасности и устранению их последствий.
	
	\subsubsection*{Защита информации направлена на:}
	\begin{itemize}
		\item  предупреждение угроз как превентивных мер по обеспечению безопасности в интересах учреждения возможности их возникновения.
		\item  выявление угроз, которое выражается в систематическом анализе и контроле возможности появления реальных и потенциальных угроз и своевременных мер по их предупреждению.
		\item  обнаружение угроз, целью которого является определение реальных угрозы или конкретных преступных деятельности.
		\item  ликвидацию последствий угроз и преступных действий и восстановления статуса-кво.
	\end{itemize}

	\textbf{\textit{Обнаружение угроз}} --- это действие по определению конкретных угроз и их источников, приносящих тот или иной вид ущерба.
	
	\textbf{\textit{Пресечение или локализация угроз}} --- это действие, направленное на устранение действующей угрозы и конкретных преступных действий 
	
	Ликвидация последствий имеет целью восстановлению состояния, предшествовавшего наступлению угрозы.
	\subsubsection{Актуальность проблемы обеспечения информационной безопасности. Задачи защиты информации}
	
	Актуальность и важность информационной безопасности(ИБ) обусловленна следующими факторами:
	\begin{itemize}
		\item высокие темпы роста парка ПК, применимых в разных сферах деятельности, и как следствие, резкое расширение круга пользователей, имеющих непосредствиный доствуп к вычислительным сетям и информационным ресурсам.
		\item увеличение объеммов информации с помощью ПК и др. средств автоматизации
		\item бурное развитие апаратно--программных средств и технологий не удовлетворяющих современных ТБ.
		\item несоответствию развития средств обработки информации и проработки теорий ИБ разработки международных стандартов и правовых норм, обеспечивающих необходимый уровень ЗИ защиты информации
		\item  повсеместное распространения сетевых технологиой, создание единого информационно-коммуникативной сети Интернет, которая не может обеспечить достойного уровня ИБ.
	\end{itemize}

	\subsubsection*{Цели защиты иформации являются}
	\begin{itemize}
		\item предотвращения утечи хищения утраты искажения поделки информации
		\item  предоствращение угроз безопасности личности, общества, гос--ва.
		\item предотвращение ненксанционированного действий по уничтожению, модификации, искажению, кописрования, блокировки информации.
		\item  предотвращение других форм незаконного вмешательства в информационные ресурсы и информационные системы
		\item  обеспечение правого режима документированной информации как объекта собственности
		\item защита конституционных прав граждан на сохранение личной тайны и конфиденциальности персональных данных, имеющихся в информационных системах
		\item сохранение государственной тайны документированной информации в соответствии с законодательством
		\item обеспечение прав субъектов в информационных процессах и при разработке, производстве и применении информационных систем, технологий и средств их обеспечения
	\end{itemize}
	\subsubsection{Задачи информационной безопасности}
	\textbf{Основные задачи системы ИБ является}
	\begin{itemize}
		\item своевременное выявление и устранение угроз безопасности и ресурсам, причин и условий, способствующей нанесению финансовой или другого ущерба.
		\item создание механизма и условий оперативного реагирования на угрозы безопасности и проявлению негативных тенденций в функционировании предприятия.
		\item эффективное пресечение посягательств на ресурсы и угроз персоналу на основе правовых, организационных и техническим мер и средств обеспечения безопасности
		\item Создание условий для возмещения м локализации нанесённого ущерба неправомерными действиями физических и юридических лиц, ослабление негативного влияния последствий нарушений безопасности на достижение целий организаций
	\end{itemize}
	
	\newpage
	\subsection{Угрозы безопасности в информационно вычислительных системах}
	Вопросы
	\begin{itemize}
		\item Понятия угрозы безопасности
		\item Актуальность проблемы обеспечение информационной безопасности
		\item Задачи защиты информации
		\item Задачи информационной безопасности
	\end{itemize}
	
	\textit{\textbf{Угроза}} --- это потенциальное возможная событие, действия(воздействия), процесс или явления которые может провести к нанесению ущерба чьим либо интересам.
	
	Существуют три разновидности угроз
	\begin{itemize}
		\item Угрозы нарушения конффендициальности
		\item Угрозы нанесения целостности
		\item Угрозы отказа служб
	\end{itemize}
	

	\textbf{\textit{Угроза нарушения конфендициальности}} --- информация становится известна тому кто не располагает полномочий к ней
	
	\textit{\textbf{Угроза нарушения целостности}} --- заключает в себе любое умышленное изменение информации хранящая в себе(вычислительной системы) или передаваемой из одной системы в другую. 
	
	\textbf{\textit{Угроза отказа служб}} --- когда в результате преднамеренных действий предпринимаемый другим пользователем или злоумышленником, блокируется доступ к некоторому ресурсу вычислительной системы.
	
	\textbf{\textit{Доступность информации}} --- это свойство системы (среды, средств и технологий  обработки) в которой циркулирует информация харктерезующая способность обеспечивать своевременную беспрепятственный доступ субъектов к интересующих их информации и готовность соответствующему автоматизированных служб к обслуживанию от субъектов запросов всегда, когда в обращении к ним.
	
	{\centering Классификация угроз информационной безопасности}
	\begin{itemize}
		\item По природе возникновения 
		\item По степени предномерености появления
		\begin{enumerate}
			\item появление ошибок в программно-апаратных средств АС
			\item некомпетентное использование или настройка или неправомерное средств защиты персоналом службы безопасности
			\item неумышленное действие, приводящая к частичному или полному отказу системы или разрушения апоратных, программынх, информационных ресурсов системы
			\item Неправомерное включение оборудования или изменение режимов работы устройств и программ
			\item Неумышленная порча носителей информации 
			\item Пересылка данных по ошибочному адресу абонента(устройства)
			\item Ввод ошибочных данных
			\item Неумышленное повреждение каналов связи 
		\end{enumerate}
		\item Угрозы преднемерного действия
		\begin{enumerate}
			\item Традиционный или универсальный шпионаж и диверсия.
			\item Несанкционированный доступ к информации 
			\item Несанкционирование модификация структур
			\item Информационные инфекции 
			\item 
		\end{enumerate}
	\end{itemize}
\end{document}