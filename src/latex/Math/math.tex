\documentclass[a4paper]{article}
\usepackage[russian]{babel}
\usepackage[utf8x]{inputenc}
\usepackage{amsmath}
\usepackage{graphicx}
\usepackage{indentfirst}

\begin{document}
	\section{Интегрирование}
	\subsection{Первообразный и неопределённый интеграл}
	
	Функция F(x) --- называется первообразной для функции f(x), на промежутке X, если x $\in$ X, выполняется равенство F'(x) = f(x)

	Тогда X, называется областью определения функции F(x)
	
	\vspace{\baselineskip}
	\begin{itshape}
		Если F(x) первообразная для функции f(x), то множество функции F(x) + C, где С произвольная постоянная, называется неопределённым интегралом от функции f(x) и обозначается 
		\begin{displaymath}
		\int f(x)\,dx=F(x)+C
		\end{displaymath}
		\vspace{\baselineskip}
		При этом функция f(x), называется под интегральной функцией.
	\end{itshape}
	
	\(\int f(x)\,dx\) (произносится как: \textit{f(x) по dx}) --- называется под интегральным выражением.
	Восстановление функции по её производной или что тоже отыскание неопределённого интеграла, называется \textbf{интегрированием}.
	
	\begin{center} \textbf{Интегрирование --- обратно дифференцированию}  \end{center}
	
	\newpage
	\begin{center}
	\textbf{Значение неопределённых интегралов}
	\end{center}
	\begin{enumerate} 
	\item	\[\int dx = x + C\]	 
	\item	\[\int x^{n}dx = \dfrac{x^{n+1}}{n+1} + C\]
	\item	\[\int\dfrac{dx}{x} = \ln{\mid x \mid} + C\]
	\item	\[\int a^{x}dx = \dfrac{a^{x}}{\ln a} + C\]
	\item	\[\int\sin x\,dx = -\cos x + C\]
	\item	\[\int\cos x\,dx = \sin x + C\]
	\item	\[\int\dfrac{dx}{\cos^{2}x} = tg x + C\]
	\item	\[\int\dfrac{dx}{\sin^{2}x} = ctg x + C\]
	\item	\[\int\dfrac{dx}{\sqrt{a^2 - x^2}} = \arcsin\dfrac{x}{a} + C\]
	\item	\[\int\dfrac{dx}{\sqrt{x^2 \pm a^2}} = \ln \mid x + \sqrt{x^2 \pm + a^2} \mid + C\]
	\item	\[\int\dfrac{dx}{a^2 + x^2} = \dfrac{1}{2a}\ln \mid\dfrac{a+x}{a-x}\mid + C\]
	\item	\[\int\dfrac{dx}{x^2 + a^2} = \dfrac{1}{2}arcctan\dfrac{x}{a} + C\]
	\item	\[\int e^x dx= e^x + C\]
	\end{enumerate}
	
	\newpage
	\begin{center}
		\textbf{Свойство интеграла}
	\end{center}
	\begin{enumerate}
		\item Производная неопределённого интеграла равна под интегральному выражению функции, а его дифференциал --- подынтегральному выражению \[ (\int f(X) dx)' = f(x), \, d(\int f(x)dx) = f(x)dx) \]
		\item Неопределённый интеграл от дифференциала функции равен сумме этих функций и произвольной константе \[ \int d\,F(x) = F(x) + C \] 
		\item Постоянный множитель можно вынести за знак неопределённого интеграла \[ \int k\,f(x)\,dx = k \int f(x)dx\]
		\item Неопределённый интеграл от суммы (разности) двух непрерывных функций равен сумме(разности) интегралов от этих функций \[ \int (f(x) \pm g(t))dx = \int f(x)dx \pm \int g(t)dx \]
	\end{enumerate}
	\newpage
	\section{Основные методы интегрирования}
	\begin{enumerate}
		\item Непосредственное интегрирование 
		\item Метод подстановки
		\item Метод интегрирования по частям 
	\end{enumerate}
	
	\subsection{Непосредственное интегрирование}
	\textbf{\textit{Непосредственное интегрирование}} --- Вычисление интегралов с помощью значений простейших неопределённых интегралов и на основе свойств неопределённых интегралов
	\subsection{Метод подстановки}
	\textbf{\textit{Метод подстановки}} --- или замена переменной заключается в том чтобы заменить x на $\phi$(t), где $\phi$(t) -- непосредственно дифференцируемая функция, полагают dx равно  $\phi$(t) * dt  и получают \[ \int f(X)dx = \int f(\phi(t)*\phi'(t))dt \]   
	\subsection{Метод интегрирования по частям}
	Формула интегрирования по частям в неопределённом интеграле называется формула:
	\[ 
		\int u\:dv = uv -\int v\:du
	\]
	\begin{itshape}
		Где \textbf{u} и \textbf{v} -- деференцируемые функции от \textbf{x}, то есть \textbf{u(x)} и \textbf{v(x)}. Формула позволяет свести вычисления интеграла\textbf{ \( \int u\:dv \)} к вычислению интеграла\textbf{ \( \int v\:du \)}, который может оказатся более простой для интегрирования.
	\end{itshape}
	
	Большую часть интегралов вычисляемых интегрированиям по частям можно разбить на 3 группы:
	\begin{enumerate}
		\item 
		\begin{enumerate}
			\item \( \int P(x) arctan(x)\:dx \)
			\item \( \int P(x) arcctg(x)\:dx \)
			\item \( \int P(x) ln(x)\:dx \)
			\item \( \int P(x) arcsin(x)\:dx \)
			\item \( \int P(x) arccos(x)\:dx \)
			\item \( \int P(x) arcctg(x)\:dx \)
		\end{enumerate}
		\begin{itshape}
			Где \textbf{P(x)} -- многочлен. Для их вычисления следует "положить"  \textbf{u} равной одной из указанной выше функции. Например в уравнении \( \int P(x) arctan(x)\:dx \) заменить \(arctan(x)\) на \textbf{u}, а деференциал равный P(x)dx.%Получаем \( \int udv\)
		\end{itshape}
		\item 
		\begin{enumerate}
			\item \( \int P(x) e^{kx}dx\)
			\item \( \int P(x) sin(kx)dx\)
			\item \( \int P(x) cos(x)dx\)
		\end{enumerate}
		\begin{itshape}
			Где P(x) -- многочлен, а k--некоторое число (может быть даже 1)  для вычисления слудует обозначить u=P(x), тогда:
		\end{itshape}
		\begin{enumerate}
			\item \( \int P(x) e^{kx}dx\quad,\quad dv = e^{kx}dx\)
			\item \( \int P(x) sin(kx)dx\quad,\quad dv = sin(kx)dx\)
			\item \( \int P(x) cos(x)dx\quad,\quad dv = cos(x)dx\)
		\end{enumerate}
		\item 
		\begin{enumerate}
			\item \( \int e^{ax}cos(bx)dx\)
			\item \( \int e^{ax}sin(bx)dx\)
		\end{enumerate}
		\begin{itshape}
			Где a и b некоторые числа, эти интегралы вычисляются двукаратным интегрированием по частям
		\end{itshape}
	\end{enumerate}
\end{document}